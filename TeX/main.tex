\documentclass{article}
\usepackage[utf8]{inputenc}
\usepackage[portuges]{babel}
\usepackage{csquotes}
\usepackage{geometry}
\usepackage[pdftex]{hyperref}
\usepackage{indentfirst}
\usepackage{amsthm}
\usepackage{amssymb}
\usepackage{amsmath}
\usepackage{indentfirst}
\usepackage[backend = biber]{biblatex}
\addbibresource{referencias.bib}

\geometry{left = 3cm, top = 3cm, bottom = 2cm, right = 2cm}

\title{Modelagem Matemática IV}
\author{Cristhian Grundmann \\
Igor Patrício Michels}
\date{2020.2}

\begin{document}

\maketitle

\section{Introdução}

O presente artigo visa descrever, por meio de um modelo epidemiológico, os casos de COVID-19 na Itália. A modelagem se dá por meio de modelos de EDO's, os quais foram vistos durante as aulas de Modelagem de Fenômenos Biológicos (Modelagem Matemática IV) ministrada pelo professor Flávio Codeço Coelho, da FGV - EMAp.

A situação italiana variou muito durante a pandemia, sendo o primeiro país europeu a atingir uma marca superior a 100 infectados pela COVID-19 após 25 dias do primeiro infectado, informação que pode ser obtida pela análise dos gráficos do site Our World in Data (veja \cite{owid}).

Em março a situação continuou se agravando, principalmente na região da Lombardia, tendo o número de infectados um crescimento exponencial. Já para os mortos em virtude da COVID-19, a Itália registrava, até 11 de março, um total de 827 mortes, sendo que os falecidos possuíam uma média de 81 anos, sendo 88,7\% destes com idade superior a 69 anos. Além disso, dois terços dos falecidos apresentavam quadro de diabetes, câncer, alguma doença cardiovascular ou fumavam. Os números continuaram subindo, sendo que no dia 15 de março os casos de COVID-19 na Itália já passavam de 22.000, com média de idade de 64 anos e 1.625 mortes, a maior parte destes tendo mais de 69 anos (veja \cite{REMUZZI20201225} e \cite{10.1001/jama.2020.4344}).


\section{Literatura}

% trabalhos de modelagem sobre a situação da Itália



\section{Metodologia}

% criar um modelo



\printbibliography

\end{document}
